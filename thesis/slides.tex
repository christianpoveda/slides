\documentclass{beamer}

\usepackage[utf8]{inputenc}
\usepackage[english]{babel}

\usepackage{amsmath}
\usepackage{nicefrac}

\usepackage{minted}
\usepackage{fontspec}

\usepackage{xcolor}

\usemintedstyle{friendly}
\setmonofont{Source Code Pro}
\usetheme{metropolis}

\newcommand{\then}{\Rightarrow}

\title{Bounded generics over constants in Rust}
\author{Author: Christian Poveda \\ Advisor: Nicolás Cardozo}
\institute{Systems and Computing Engineering Department \\ Universidad de los Andes}
\date{2018-09-18}

\begin{document}

\frame{\titlepage}

\begin{frame}[fragile]
    \frametitle{What is a type?}
    Que es un tipo para un programador
\end{frame}

\begin{frame}[fragile]
    \frametitle{What is a type?}
    Que es un sitema de tipos para alguien que trabaja con sistemas de tipos
\end{frame}

\begin{frame}[fragile]
    \frametitle{Context: The Rust programming language}
    Que es Rust, que lo hace distinto de otros lenguajes
\end{frame}

\begin{frame}[fragile]
    \frametitle{Context: Rust's type system}
    Que sistema de tipos tiene Rust y que le permite hacer
\end{frame}

\begin{frame}[fragile]
    \frametitle{The problem: Traits over arrays}
    Rust no deja implementar traits sin tamaños especificos de arrays
\end{frame}

\begin{frame}[fragile]
    \frametitle{The problem: Some example with bounds}
\end{frame}

\begin{frame}[fragile]
    \frametitle{The solution: Constant values as type parameters}
    Presentar const-generics 
\end{frame}

\begin{frame}[fragile]
    \frametitle{The solution: Bounds over constant values}
    Presentar como poner bounds para const-generics 
\end{frame}

\begin{frame}[fragile]
    \frametitle{The result: Arrays as const-generic types}
    Con const generics, podemos implementar traits para tamaños genericos de arrays
\end{frame}

\begin{frame}[fragile]
    \frametitle{The result: Solution to the bounds example}
\end{frame}

\begin{frame}[fragile]
    \frametitle{Context: Generics over values in theory}
    Circulos
\end{frame}

\begin{frame}[fragile]
    \frametitle{Context: Languages with dependent types}
    Haskell, Idris, Agda, Coq, Otros. Mostrar número de papers en el último año para cada uno
\end{frame}

\begin{frame}[fragile]
    \frametitle{Context: Idris, a dependently typed language}
    Como se ve Idris y que deja hacer
\end{frame}

\begin{frame}[fragile]
    \frametitle{Context: Rust Status Quo}
    RFC-2000, que hay, que habrá y que quedará faltando
\end{frame}

\begin{frame}[fragile]
    \frametitle{Road Ahead: What needs to be done}
    Desbaratar los ejemplos para mostrar que pasos hay que seguir, entre ellos unificación
\end{frame}

\begin{frame}[fragile]
    \frametitle{Road Ahead: Unification}
    Mostrar que alternativas hay para implementarla
\end{frame}

\begin{frame}[fragile]
    \frametitle{Validation: Formal verification}
    Proveer una prueba formal de que unificación y bounds son bien comportados 
\end{frame}

\begin{frame}[fragile]
    \frametitle{Validation: Comparison against other languages}
    Con estas nuevas features hasta donde puede dar Rust al compararlo con lenguajes como Idris
\end{frame}

\begin{frame}[fragile]
    \frametitle{Validation: Integration with Rust}
    Integrar este trabajo dentro de Rust como tal, RFCs y PRs al respecto
\end{frame}

\begin{frame}[fragile]
    \frametitle{Schedule}
    El cronograma... para (no) cumplirlo
\end{frame}
