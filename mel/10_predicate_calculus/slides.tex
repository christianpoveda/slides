\documentclass{beamer}

\usepackage[utf8]{inputenc}
\usepackage[english]{babel}

\usepackage{amsmath}
\usepackage{nicefrac}

\usepackage{minted}
\usepackage{fontspec}

\usepackage{xcolor}

\usemintedstyle{friendly}
\setmonofont{Source Code Pro}
\usetheme{metropolis}

\newcommand{\then}{\Rightarrow}

\title{Cálculo de predicados}
\author{Matemática estructural y lógica}
\institute{ISIS-1104}
\date{}

\begin{document}

\frame{\titlepage}

\begin{frame}[fragile]
    \frametitle{Axiomas de la lógica de predicados}
    \textbf{Trueque:}
    $$(\forall x \mid R : P) \equiv (\forall x \mid : R \then P)$$
    En otras palabras
    \begin{center}
        \textit{"Para todo x tal que $R$ se tiene $P$"} \\
        es lo mismo que \\
        \textit{"Para todo x se tiene que si $R$ entonces $P$"}
    \end{center}
\end{frame}

\begin{frame}[fragile]
    \frametitle{Axiomas de la lógica de predicados}
    \textbf{Distributividad entre $\lor$ y $\forall$:} \\
    Si $P$ no depende de $x$
    $$P \lor (\forall x \mid R : Q) \equiv (\forall x \mid R : P \lor Q)$$
\end{frame}

\begin{frame}[fragile]
    \frametitle{Axiomas de la lógica de predicados}
    \textbf{Ley de De Morgan generalizada:} \\
    $$\lnot (\exists x \mid Q : R) \equiv (\forall x \mid Q : \lnot R)$$
    En otras palabras
    \begin{center}
        \textit{"Negar la existencia de un x que cumple $R$} \\
        es lo mismo que \\
        \textit{"Afirmar que para todo x se cumple que $\lnot R$"}
    \end{center}
    Es mas útil escribir esta regla como
    $$(\exists x \mid Q : R) \equiv \lnot (\forall x \mid Q : \lnot R)$$
\end{frame}

\begin{frame}[fragile]
    \frametitle{Una demostración en lógica de predicados}
    \textbf{Teorema:}
    $$(\forall x \mid R : true) \equiv true$$
    \textbf{Demostración:} \\
    Por idempotencia $true \equiv true \lor true$, luego
    $$(\forall x \mid R : true) \equiv (\forall x \mid R : true \lor true)$$
    Usando distribución entre $\lor$ y $\forall$
    $$(\forall x \mid R : true \lor true) \equiv true \lor (\forall x \mid R : true)$$
    Usando techo
    $$true \lor (\forall x \mid R : true) \equiv true$$
\end{frame}

\begin{frame}[fragile]
    \frametitle{Otra demostración en lógica de predicados}
    \textbf{Teorema (Trueque para $\exists$):}
    $$(\exists x \mid R : P) \equiv (\exists x \mid : R \land P)$$
    \textbf{Demostración:} \\
    \begin{align*}
        (\exists x \mid R : P) &\equiv \lnot (\forall x \mid R : \lnot P) \tag{De Morgan gen.}\\ 
        &\equiv \lnot (\forall x \mid : R \then \lnot P) \tag{trueque para $\forall$}\\
        &\equiv \lnot (\forall x \mid : \lnot R \lor \lnot P) \tag{definición $\then$}\\
        &\equiv \lnot (\forall x \mid : \lnot (R \land P)) \tag{De Morgan }\\
        &\equiv (\exists x \mid : R \land P) \tag{De Morgan gen.}\\
    \end{align*}
\end{frame}

\begin{frame}[fragile]
    \frametitle{Ahora ustedes}
    \textbf{Teorema (Distribución entre $\land$ y $\exists$):} \\
    Si $P$ no depende de $x$:
    $$P \land (\exists x \mid R : Q) \equiv (\exists x \mid : P \land Q)$$
\end{frame}

\begin{frame}[fragile]
    \frametitle{Ahora ustedes}
    \textbf{Teorema:} \\
    $$(\exists x \mid R : false) \equiv false$$
\end{frame}

\end{document}
