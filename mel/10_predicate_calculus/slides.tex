\documentclass{beamer}

\usepackage[utf8]{inputenc}
\usepackage[english]{babel}

\usepackage{amsmath}
\usepackage{nicefrac}

\usepackage{minted}
\usepackage{fontspec}

\usepackage{xcolor}

\usemintedstyle{friendly}
\setmonofont{Source Code Pro}
\usetheme{metropolis}

\newcommand{\then}{\Rightarrow}

\title{Cálculo de predicados}
\author{Matemática estructural y lógica}
\institute{ISIS-1104}
\date{}

\begin{document}

\frame{\titlepage}

\begin{frame}[fragile]
    \frametitle{Axiomas de la lógica de predicados}
    \pause
    \textbf{Trueque:}
    $$(\forall x \mid R : P) \equiv (\forall x \mid : R \then P)$$
    \pause
    En otras palabras
    \begin{center}
        \textit{"Para todo x tal que $R$ se tiene $P$"} \\
        es lo mismo que \\
        \textit{"Para todo x se tiene que si $R$ entonces $P$"}
    \end{center}
\end{frame}

\begin{frame}[fragile]
    \frametitle{Axiomas de la lógica de predicados}
    \pause
    \textbf{Distributividad entre $\lor$ y $\forall$:} \\
    Si $P$ no depende de $x$
    $$P \lor (\forall x \mid R : Q) \equiv (\forall x \mid R : P \lor Q)$$
\end{frame}

\begin{frame}[fragile]
    \frametitle{Axiomas de la lógica de predicados}
    \pause
    \textbf{Ley de De Morgan generalizada:} \\
    $$\lnot (\exists x \mid Q : R) \equiv (\forall x \mid Q : \lnot R)$$
    \pause
    En otras palabras
    \begin{center}
        \textit{"Negar la existencia de un x que cumple $R$} \\
        es lo mismo que \\
        \textit{"Afirmar que para todo x se cumple que $\lnot R$"}
    \end{center}
    \pause
    Es mas útil escribir éste axioma como
    $$(\exists x \mid Q : R) \equiv \lnot (\forall x \mid Q : \lnot R)$$
\end{frame}

\begin{frame}[fragile]
    \frametitle{Una demostración en lógica de predicados}
    \pause
    \textbf{Teorema:}
    $$(\forall x \mid R : true) \equiv true$$
    \pause
    \textbf{Demostración:} \\
    \pause
    Por idempotencia $true \equiv true \lor true$, luego
    $$(\forall x \mid R : true) \equiv (\forall x \mid R : true \lor true)$$
    \pause
    Usando distribución entre $\lor$ y $\forall$
    $$(\forall x \mid R : true \lor true) \equiv true \lor (\forall x \mid R : true)$$
    \pause
    Usando techo
    $$true \lor (\forall x \mid R : true) \equiv true$$
\end{frame}

\begin{frame}[fragile]
    \frametitle{Otra demostración en lógica de predicados}
    \pause
    \textbf{Teorema (Trueque para $\exists$):}
    $$(\exists x \mid R : P) \equiv (\exists x \mid : R \land P)$$
    \pause
    \textbf{Demostración:} \\
    \pause
    \begin{align*}
        (\exists x \mid R : P) &\equiv \lnot (\forall x \mid R : \lnot P) \tag{De Morgan gen.}\\ 
        &\equiv \lnot (\forall x \mid : R \then \lnot P) \tag{trueque para $\forall$}\\
        &\equiv \lnot (\forall x \mid : \lnot R \lor \lnot P) \tag{definición $\then$}\\
        &\equiv \lnot (\forall x \mid : \lnot (R \land P)) \tag{De Morgan }\\
        &\equiv (\exists x \mid : R \land P) \tag{De Morgan gen.}\\
    \end{align*}
\end{frame}

\begin{frame}[fragile]
    \frametitle{Ahora ustedes}
    \pause
    \textbf{Teorema (Distribución entre $\land$ y $\exists$):} \\
    Si $P$ no depende de $x$:
    $$P \land (\exists x \mid R : Q) \equiv (\exists x \mid R : P \land Q)$$
    \vspace*{165 pt}
\end{frame}

\begin{frame}[fragile]
    \frametitle{Ahora ustedes}
    \pause
    \textbf{Teorema:} \\
    $$(\exists x \mid R : false) \equiv false$$
    \vspace*{165 pt}
\end{frame}

\begin{frame}[fragile]
    \frametitle{Ahora ustedes}
    \pause
    \textbf{Teorema:} \\
    Si $P$ no depende de $x$:
    $$Q \land (\exists x \mid : P) \equiv (\exists x \mid P : Q)$$
    \vspace*{165 pt}
\end{frame}

\begin{frame}[fragile]
    \frametitle{Reglas de inferencia: Eliminación}
    Cuando razonamos de forma intuitiva, recurrimos a los siguientes argumentos:
    \begin{center}
        \textit{"Todos los perros son juiciosos, en particular Milou es juicioso."}

        \vspace{10pt}

        \textit{"Existen números interesantes, luego sea $k$ un número interesante."}
    \end{center}
    En otras palabras estamos \textbf{eliminando} cuantificaciones.
\end{frame}

\begin{frame}[fragile]
    \frametitle{Reglas de inferencia: Eliminación}
    Cuando razonamos de forma intuitiva, recurrimos a los siguientes argumentos:
    \begin{center}
        $(\forall p \mid : juicioso(p)) \then juicioso(\text{milou})$

        \vspace{10pt}

        $(\exists n \mid : interesante(n)) \then interesante(k)$
    \end{center}
    En otras palabras estamos \textbf{eliminando} cuantificaciones.
\end{frame}

\begin{frame}[fragile]
    \frametitle{Reglas de inferencia: Eliminación}
    \textbf{Eliminación $\forall$:} Si tenemos que $\big(\forall x \mid : P(x)\big)$, podemos asegurar que \textbf{cualquier} constante cumple $P$ 
        $$\frac{\big(\forall x \mid : P(x)\big)} {P(c)\ \text{\footnotesize(cualquier $c$)}}$$
    \textbf{Eliminación $\exists$:} Si tenemos que $\big(\exists x \mid : P(x)\big)$, podemos \textbf{introducir} una constante nueva que cumpla $P$
    $$\frac{\big(\exists x \mid : P(x)\big)} {P(c)\ \text{\footnotesize(nueva $c$)}}$$
    Se ven muy parecidas, pero la diferencia es grande.
\end{frame}

\begin{frame}[fragile]
    \frametitle{Reglas de inferencia: Introducción}
    Cuando razonamos de forma intuitiva, recurrimos a los siguientes argumentos:
    \begin{center}
        \textit{"Messi es argentino. Por lo tanto, existen futbolistas argentinos."}

        \vspace{10pt}

        \textit{"Un futbolista es rápido. Luego cualquier futbolista es rápido."}

    \end{center}
    En otras palabras estamos \textbf{introduciendo} cuantificaciones.
\end{frame}

\begin{frame}[fragile]
    \frametitle{Reglas de inferencia: Introducción}
    Cuando razonamos de forma intuitiva, recurrimos a los siguientes argumentos:
    \begin{center}
        $argentino(\text{messi}) \then \big(\exists j \mid : argentino(j)\big)$

        \vspace{10pt}

        $ rapido(j) \then \big(\forall j \mid : rapido(j)\big)$
    \end{center}
    En otras palabras estamos \textbf{introduciendo} cuantificaciones.
\end{frame}

\begin{frame}[fragile]
    \frametitle{Reglas de inferencia: Eliminación}
    \textbf{Introducción $\exists$:} Si tenemos que $P(c)$, donde $c$ es una constante que \textbf{ya habíamos introducido}:
    $$\frac {P(c)\ \text{\footnotesize($c$ ya introducida)}}{\big(\exists x \mid : P(x)\big)} $$
    \textbf{Introducción $\forall$:} Si demostramos $P(c)$, donde $c$ es una constante \textbf{arbitraria}: 
        $$\frac{P(c)\ \text{\footnotesize($c$ arbitraria)}}{\big(\forall x \mid : P(x)\big)}$$
    Se ven muy parecidas, pero la diferencia es grande.
\end{frame}

\begin{frame}[fragile]
    \frametitle{Reglas de inferencia: Ejemplo}
    \textbf{Teorema (súper Modus Ponens):}
    $$\frac{\big(\forall x \mid P(x): Q(x)\big) \land P(c)}{Q(c)}$$
    \textbf{Demostración:} \\
    Usando trueque
    $$\frac{\big(\forall x \mid P(x): Q(x)\big)}{\big(\forall x \mid : P(x) \then Q(x)\big)}$$
    Usando eliminación 
    $$\frac{\big(\forall x \mid : P(x) \then Q(x)\big)}{P(c) \then Q(c)}$$
    Usando Modus Ponens
    $$\frac{P(c)\land (P(c) \then Q(c))}{Q(c)}$$
\end{frame}

\begin{frame}[fragile]
    \frametitle{Reglas de inferencia: Ejercicio}
    \textbf{Teorema (súper Silogismo Hipotético):}
    $$\frac{\big(\forall x \mid P(x): Q(x)\big) \land (Q(c) \then R(c))}{P(c) \then R(c)}$$
    \vspace{160pt}
\end{frame}

\begin{frame}[fragile]
    \frametitle{Reglas de inferencia: Ejemplo}
    \textbf{Teorema (debilitamiento del rango):}
    $$\frac{(\forall x \mid P(x) \lor Q(x): R(x))}{(\forall x \mid P(x): R(x))} \hspace{20pt} \frac{(\exists x \mid P(x): R(x))}{(\forall x \mid P(x) \lor Q(x): R(x))}$$
    \vspace{160pt}
\end{frame}

\begin{frame}[fragile]
    \frametitle{Reglas de inferencia: Ejemplo}
    \textbf{Teorema (debilitamiento del cuerpo):}
    $$\frac{(\forall x \mid P(x) : Q(x) \land R(x))}{(\forall x \mid P(x): Q(x))} \hspace{20pt} \frac{(\exists x \mid P(x): R(x))}{(\exists x \mid P(x) : Q(x) \lor R(x))}$$
    \vspace*{165 pt}
\end{frame}

\end{document}
