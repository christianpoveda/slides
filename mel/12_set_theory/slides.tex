\documentclass{beamer}

\usepackage[utf8]{inputenc}
\usepackage[english]{babel}

\usepackage{amsmath}
\usepackage{nicefrac}

\usepackage{minted}
\usepackage{fontspec}

\usepackage{xcolor}

\usemintedstyle{friendly}
\setmonofont{Source Code Pro}
\usetheme{metropolis}

\newcommand{\then}{\Rightarrow}

\title{Teoría de conjuntos}
\author{Matemática estructural y lógica}
\institute{ISIS-1104}
\date{}

\begin{document}

\frame{\titlepage}

\begin{frame}[fragile]
    \frametitle{Reglas}
    \begin{itemize}
        \item Un conjunto es una colección de elementos no repetidos.
        \item Dos conjuntos son iguales si tienen los mismos elementos.
        \item Un conjunto puede pertenecer a otro conjunto.
        \item Ningún conjunto se contiene a si mismo.
    \end{itemize}
\end{frame}

\begin{frame}[fragile]
    \frametitle{¿Cómo denotar conjuntos?}
    \textbf{Notación extensional:} Simplemente escribimos todos los elementos dentro de
    llaves
    $$A = \{0, 1, 2, 3, 4\}$$
    $$B = \{2, 4, 6, 8, 10, ...\}$$
    \textbf{Notación intensional:} Escribimos algun predicado que los elementos del conjunto deben cumplir
    $$A = \{x:\text{int} \mid x \geq 0 \land x < 5 \}$$
    $$B = \{x:\text{int} \mid x > 0 \land x \text{ es par}\}$$    
\end{frame}

\begin{frame}[fragile]
    \frametitle{Pertenencia y contenecia}
    Para afirmar que un elemento $x$ pertenece al conjunto $A$, escribimos el predicado
    $$ x \in A$$
    Podemos definir la igualdad de conjuntos utilizando pertenencia 
    $$A = B \equiv (\forall x \mid : x \in A \equiv x \in B)$$
    También podemos definir la contenencia de conjuntos
    $$A \subseteq B \equiv (\forall x \mid : x \in A \then x \in B)$$
\end{frame}

\end{document}
