\documentclass{beamer}

\usepackage[utf8]{inputenc}
\usepackage[english]{babel}

\usepackage{amsmath}
\usepackage{nicefrac}

\usepackage{minted}
\usepackage{fontspec}

\usepackage{xcolor}

\usemintedstyle{friendly}
\setmonofont{Source Code Pro}
\usetheme{metropolis}

\newcommand{\then}{\Rightarrow}

\title{Teoría de conjuntos}
\author{Matemática estructural y lógica}
\institute{ISIS-1104}
\date{}

\begin{document}

\frame{\titlepage}

\begin{frame}[fragile]
    \frametitle{Reglas}
    \begin{itemize}
        \item Un conjunto es una colección de elementos no repetidos.
        \item Dos conjuntos son iguales si tienen los mismos elementos.
        \item Un conjunto puede pertenecer a otro conjunto.
        \item Ningún conjunto pertenece a si mismo.
    \end{itemize}
\end{frame}

\begin{frame}[fragile]
    \frametitle{¿Cómo denotar conjuntos?}
    \textbf{Notación extensional:} Simplemente escribimos todos los elementos dentro de
    llaves
    $$A = \{0, 1, 2, 3, 4\}$$
    $$B = \{2, 4, 6, 8, 10, ...\}$$
    \textbf{Notación intensional:} Escribimos algun predicado que los elementos del conjunto deben cumplir
    $$A = \{x:\text{int} \mid x \geq 0 \land x < 5 \}$$
    $$B = \{x:\text{int} \mid x > 0 \land x \text{ es par}\}$$    
\end{frame}

\begin{frame}[fragile]
    \frametitle{Pertenencia}
    Para afirmar que un elemento $x$ pertenece al conjunto $A$, escribimos el predicado
    $$x \in A\ \text{ o }\ x : A$$
    Para negar pertenencia escribimos 
    $$\lnot (x \in A)\ \text{ o }\ x \not \in A$$
    Dado que la notación intensional es de la forma
    $$\{x \mid P(x)\}$$
    Es natural afirmar que
    $$(y \in \{x \mid P(x)\}) \equiv P(y)$$
\end{frame}

\begin{frame}[fragile]
    \frametitle{Igualdad y contenencia}
    Podemos definir la igualdad de conjuntos utilizando pertenencia 
    $$(A = B) \equiv (\forall x \mid : x \in A \equiv x \in B)$$
    También podemos definir la contenencia de conjuntos
    $$(A \subseteq B) \equiv (\forall x \mid : x \in A \then x \in B)$$
    \textbf{Teorema:} $(A = B) \equiv (A \subseteq B) \land (B \subseteq A)$
\end{frame}

\begin{frame}[fragile]
    \frametitle{El universo y el vacío}
    \textbf{Conjunto vacío:} Lo denotamos con el simbolo $\varnothing$, es el conjunto que no tiene elementos
    $$(\forall x \mid : x \not \in \varnothing)$$
    \textbf{Conjunto universo:} Lo denotamos con la letra $\mathcal{U}$, es el conjunto que contiene "todos" los elementos
    $$(\forall x \mid x \neq \mathcal{U}: x \in \mathcal{U})$$
    \textbf{Teorema:} Para cualquier conjunto $A$ se tiene que
    $$\varnothing \subseteq A$$
    $$A \subseteq \mathcal{U} \hspace{10pt} \text{(cuando $A \neq \mathcal{U}$)}$$
\end{frame}

\begin{frame}[fragile]
    \frametitle{Operaciones entre conjuntos}
    Definimos las siguientes operaciones
    \begin{itemize}
        \item Unión de conjuntos
        $$A \cup B = \{x \mid (x \in A) \lor (x \in B)\}$$
        \item Intersección de conjuntos
        $$A \cap B = \{x \mid (x \in A) \land (x \in B)\}$$
        \item Complemento de un conjunto
        $$A^{c} = \{x \mid \lnot(x \in A)\}$$
        \item Producto cartesiano de conjuntos
        $$A \times B = \{(x, y) \mid (x \in A) \land (x \in B)\}$$
    \end{itemize}
\end{frame}

\begin{frame}[fragile]
    \frametitle{Operaciones entre conjuntos}
    O equivalentemente
    \begin{itemize}
        \item Unión de conjuntos
        $$(x \in A \cup B) \equiv (x \in A) \lor (x \in B)$$
        \item Intersección de conjuntos
        $$(x \in A \cap B) \equiv (x \in A) \land (x \in B)$$
        \item Complemento de un conjunto
        $$(x \in A ^c) \equiv \lnot (x \in A)$$
        \item Producto cartesiano de conjuntos
            $$((x, y) \in A \times B) \equiv (x \in A) \land (x \in B)$$
    \end{itemize}
\end{frame}

\begin{frame}[fragile]
    \frametitle{Teoría de conjuntos y lógica}
    En teoría de conjuntos son válidos los siguientes teoremas
    \begin{align*}
        A \cap B &= B \cap A \\ 
        A \cap (B \cap C) &= (A \cap B) \cap C \\
        A \cap (A \cup B) &= A \\
        A \cap (B \cup C) &= (A \cap B) \cup (A \cap C) \\
        A \cap A^{C} &= \varnothing \\
        A \cap \mathcal{U} &= A \\
        A \cap \varnothing &= \varnothing \\
        A \cap A &= A \\
        (A^C)^C &= A \\
        (A \cap B)^c &= A^c \cup B^c \\
    \end{align*}
\end{frame}

\begin{frame}[fragile]
    \frametitle{Teoría de conjuntos y lógica}
    Porque en realidad son los mísmos axiomas de lógica proposicional
    \begin{align*}
        p \land q &\equiv q \land p \\
        p \land (q \land r) &\equiv (p \land q) \land r \\
        p \land (p \lor q) &\equiv p \\
        p \land (q \lor r) &\equiv (p \land q) \lor (p \land r) \\
        p \land \lnot p &\equiv False \\
        p \land True &\equiv p \\
        p \land False &\equiv False \\
        p \land p &\equiv p \\
        \lnot \lnot p &\equiv p \\
        \lnot (p \land q) &\equiv \lnot p \lor \lnot q \\
    \end{align*}
\end{frame}

\begin{frame}[fragile]
    \frametitle{Un ejemplo}
    Demostremos que para cualquier conjunto $A$ se tiene que
    $$A \cap A^C = \varnothing$$
    Por definición de igualdad
    $$(\forall x \mid : (x \in A \cap A^c) \equiv (x \in \varnothing))$$
    Por $\forall$-eliminación
    $$(x \in A \cap A^c) \equiv (x \in \varnothing)$$
    Entonces partímos de $x \in A \cap A^c$ e intentamos llegar a $x \in \varnothing$
\end{frame}

\begin{frame}[fragile]
    \frametitle{Un ejemplo}
    Por definición de $\cap$
    $$(x \in A \cap A^c) \equiv (x \in A) \land (x \in A^c)$$
    Por definición de complemento
    $$(x \in A) \land (x \in A^c) \equiv (x \in A) \land \lnot (x \in A)$$
    Por \textbf{Piso}
    $$(x \in A) \land \lnot (x \in A) \equiv False$$
    Sin embargo, por definición de vacío tenemos que $\lnot (x  \in \varnothing)$, o equivalentemente
    $$False \equiv (x \in \varnothing)$$
    Luego
    $$(x \in A \cap A^c) \equiv (x \in \varnothing)$$
\end{frame}

\begin{frame}[fragile]
    \frametitle{Ahora ustedes}
    Demuestren que para cualquier conjunto $A$
    $$A \cup \varnothing = A$$
    \vspace{160pt}
\end{frame}

\begin{frame}[fragile]
    \frametitle{¿Por qué esto funciona?}
    Nuestras demostraciones consistieron en mostrar que
    $$A \cap A^c = \varnothing \text{ se podía convertir a } (x \in A) \land \lnot (x \in A) \equiv False$$
    $$A \cup \varnothing = A \text{ se podía convertir a } (x \in A) \lor False \equiv (x \in A)$$
    Esto \textbf{siempre} puede hacerse para cualquier afirmación sobre conjuntos y se conoce con el nombre de \textbf{teorema de representación}.
\end{frame}

\begin{frame}[fragile]
    \frametitle{El teorema de representación}
    No tenemos el lenguaje adecuado para describir este teorema con exactitud,
    pero su enunciado sería algo así:

    \begin{center}
        Cualquier afirmación sobre conjuntos puede transformarse a un predicado sustituyendo
        \begin{itemize}
            \item "$\cap$" por "$\land$"
            \item "$\cup$" por "$\land$"
            \item "$c$" por "$\lnot$"
            \item "$=$" por "$\equiv$"
            \item "$\varnothing$" por "$False$"
            \item "$\mathcal{U}$" por "$True$"
            \item Cada conjunto "$A$" por el predicado "$x \in A$"
        \end{itemize}
    \end{center}
\end{frame}

\begin{frame}[fragile]
    \frametitle{Un ejemplo: De Morgan para conjuntos}
    \textbf{Teorema:}
    $$(A \cup B)^c = A^c \cap B^c$$
    \textbf{Demostración:}
    Usando el teorema de representación
    $$\lnot((x \in A) \lor (x \in B)) \equiv \lnot(x \in A) \land \lnot(x \in B)$$
    Lo cual es trivial de demostrar usando De Morgan para proposiciones.
\end{frame}

\begin{frame}[fragile]
    \frametitle{Ahora ustedes}
    \textbf{Teorema:}
    $$A \cap (A \cup B) = A$$
    \vspace{165pt}
\end{frame}



\end{document}

