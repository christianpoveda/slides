\documentclass{beamer}

\usepackage[utf8]{inputenc}
\usepackage[english]{babel}

\usepackage{amsmath}
\usepackage{nicefrac}

\usepackage{minted}
\usepackage{fontspec}

\usepackage{xcolor}

\usemintedstyle{friendly}
\setmonofont{Source Code Pro}
\usetheme{metropolis}

\newcommand{\then}{\Rightarrow}

\title{Cuantificadores}
\author{Matemática estructural y lógica}
\institute{ISIS-1104}
\date{}

\begin{document}

\frame{\titlepage}

\begin{frame}[fragile]
    \frametitle{Operadores binarios}
        Un operador binario sobre elementos de tipo $T$ es una función $T \times T \rightarrow T$
            $$+:Int \times Int \rightarrow Int$$.
        Decimos que un operador $\oplus:T \times T \rightarrow T$ es: 
        \begin{itemize}
            \item Asociativo si $(a \oplus b) \oplus c = a \oplus (b \oplus c)$
            \item Conmutativo si $(a \oplus b) = (b \oplus a)$
            \item Con $n$ como identidad si $(a \oplus n) = (n \oplus a) = a$
    \end{itemize}
\end{frame}

\begin{frame}[fragile]
    \frametitle{Algunos ejemplos}
    \begin{itemize}
        \item $+$ en $Int$, es asociativo, conmutativo y con $0$ como identidad.
        \item $/$ en $Real$, es no asociativo, no conmutativo y sin identidad.
        \item $\lor$ en $Bool$, es asociativo, conmutativo y con $False$ como identidad.
        \item $\equiv$ en $Bool$, es asociativo, conmutativo y con $True$ como identidad.
    \end{itemize}
\end{frame}

\begin{frame}[fragile]
    \frametitle{Ahora ustedes!}
    \begin{itemize}
        \item $-$ en $Int$?
        \item $*$ en $Int$?
        \item $\land$ en $Bool$?
        \item $\then$ en $Bool$?
    \end{itemize}
\end{frame}

\begin{frame}[fragile]
    \frametitle{Cuantificadores}
    Si $\oplus:T \rightarrow T \times T$ es conmutativo, asociativo y con identidad. Definimos su cuantificación como
        $$(\oplus t:T \mid Q:E)$$
    Lo cual se interpreta como
        \begin{center}
            \it{sobre todos los $t$ que cumplan $Q(t)$,\\ agregar el resultado de $E(t)$ usando $\oplus$}
        \end{center}
    $Q:T \rightarrow Bool$ se conoce como el \it rango.
    
    $E:T \rightarrow T$ se conoce como el \it cuerpo.
\end{frame}


\begin{frame}[fragile]
    \frametitle{Cuantificadores vs. funciones de alto orden}
    En realidad un cuantificador es una combinación de funciones de alto orden
        $$(\oplus t:T \mid Q:E)$$
    es lo mismo que
    \begin{minted}[escapeinside=||, autogobble]{java}
        static T quantifier(|$\oplus$|, |$Q$|, |$E$|, list) {
            List<T> filtered = filter(|$Q$|, list);
            List<T> mapped = map(|$E$|, filtered);
            T folded = fold(|$\oplus$|, mapped);
            return folded;
        } 
    \end{minted}
    Un cuantificador aplica sobre todos los elementos de tipo $T$  mientras que \verb|quantifier| no.
\end{frame}

\begin{frame}[fragile]
    \frametitle{Un ejemplo}
    $$(+ i:Int \mid i \geq 0 \land i < 6: 2i)$$
    Podemos pensarlo como
    \begin{itemize}
        \item Filtramos: Nos quedamos con los $i$ que cumplan $i \geq 0 \land i < 6$.
            $$[0, 1, 2, 3, 4, 5]$$
        \item Mapeamos: Transformamos estos $i$ a $2i$
            $$[0, 2, 4, 6, 8, 10]$$
        \item Reducimos: Operamos usando $+$
            $$0 + 2 + 4 + 6 + 8 + 10 = 30$$
    \end{itemize}
\end{frame}

\begin{frame}[fragile]
    \frametitle{Ahora ustedes}
    $$(* i:Int \mid i > 0 \land i < 10 \land esPar(i): i/2)$$
    \vspace*{165 pt}
\end{frame}

\begin{frame}[fragile]  
    \frametitle{Algunos casos particulares}
    \begin{itemize}
    \item Si no queremos filtrar hacemos el rango siempre verdadero:
        $$(\oplus t:T \mid True:E)\text{ o más corto }(\oplus t:T \mid:E)$$
    \item Las cuantificaciones pueden depender de otros valores:
        $$(* i:Int \mid i>0 \land i < n :i)$$
    \item Podemos operar infinitos elementos
        $$(+ i:Int \mid i > 0:\nicefrac{1}{2^{i}}) = \frac{1}{2^{1}} + \frac{1}{2^{2}} + \frac{1}{2^{3}} + ... = 2$$
    \item Pero no siempre tenemos que obtener un valor finito
        $$(+ i:Int \mid i > 0: i) \text{ diverge }$$
    \end{itemize}
\end{frame}

\begin{frame}[fragile]
    \frametitle{¿Cómo evaluar cuantificadores? Axiomas}
    \begin{itemize}
        \item Rango vacío: 
            $$(\oplus t \mid False: E) = \text{identidad de }\oplus$$
        \item Regla de un punto:
            $$(\oplus t \mid t = s: E) = E(s)$$
        \item Distributividad$^{*}$:
            $$(\oplus t \mid Q: E_{1})\ \oplus\ (\oplus t \mid Q: E_{2}) = (\oplus t \mid Q: E_1 \oplus E_2)$$
    \end{itemize}
\end{frame}

\begin{frame}[fragile]
    \frametitle{¿Cómo evaluar cuantificadores? Axiomas}
    \begin{itemize}
        \item Partir rango disyunto$^{*}$: Si $Q \land S \equiv False$ para todos los $t$
            $$(\oplus t \mid Q \lor S : E) = (\oplus t \mid Q : E)\ \oplus\ (\oplus t \mid S : E)$$
        \item Partir rango idempotente$^{*}$: Si $t \oplus t = t$ para todos los $t$
            $$(\oplus t \mid Q \lor S : E) = (\oplus t \mid Q : E)\ \oplus\ (\oplus t \mid S : E)$$
        \item Partir rango general$^{*}$:
            $$(\oplus t \mid Q \lor S : E)\ \oplus\ (\oplus t \mid Q \land S : E) = (\oplus t \mid Q : E)\ \oplus\ (\oplus t \mid S : E)$$
    \end{itemize}
\end{frame}

\begin{frame}[fragile]
    \frametitle{¿Cómo evaluar cuantificadores? Axiomas}
    \begin{itemize}
        \item Intercambio de variables$^{*}$: Si $Q$ no depende de $v$ y $S$ no depende de $t$
            $$(\oplus t \mid Q : (\oplus v \mid S : E)) = (\oplus v \mid S : (\oplus t \mid Q : E))$$
        \item Anidamiento$^{*}$: Si $Q$ no depende de $v$
            $$(\oplus t,v \mid Q \land S : E) = (\oplus t \mid Q : (\oplus v \mid S : E))$$
        \item Renombramiento de variables: Si $Q$ y $E$ no dependen de $v$
            $$(\oplus t \mid Q : E) = (\oplus v \mid Q : E)$$
    \end{itemize}
\end{frame}

\end{document}
