\documentclass{beamer}

\usepackage[utf8]{inputenc}
\usepackage[english]{babel}

\usepackage{amsmath}
\usepackage{nicefrac}

\usepackage{minted}
\usepackage{fontspec}

\usepackage{xcolor}

\usemintedstyle{friendly}
\setmonofont{Source Code Pro}
\usetheme{metropolis}

\newcommand{\then}{\Rightarrow}

\title{Traducciones de lenguaje natural}
\author{Matemática estructural y lógica}
\institute{ISIS-1104}
\date{}

\begin{document}

\frame{\titlepage}

\begin{frame}[fragile]
    \frametitle{Un ejemplo}
    \begin{center}
        \textit{Ningún superheroe distinto de Thor, \\
        es digno de levantar el martillo. \\
        Hulk esta alzando a Thor. \\
        Thor es digno de levantar el martillo. \\
        Luego, Hulk es digno de levantar el martillo.}
    \end{center}
\end{frame}

\begin{frame}[fragile]
    \frametitle{Un ejemplo}
    \pause
    Para traducir una afirmación a lógica de predicados se debe:
    \begin{itemize}
        \pause
        \item Establecer el tipo de nuestras variables.
        \pause
        \item Extraer las constantes de nuestra afirmación.
        \pause
        \item Traducir los predicados más básicos de nuestra afirmación.
        \pause
        \item Traducir las premisas y la conclusión
        \pause
        \item Unir las premisas y conclusión
    \end{itemize}
\end{frame}

\begin{frame}[fragile]
    \frametitle{Un ejemplo}
    \begin{itemize}
        \pause
        \item Todas las variables son de tipo $S$ o superhéroe.
        \pause
        \item Constantes:
            \begin{align*}
                t : S &= \text{el superhéroe Thor} \\
                h : S &= \text{el superhéroe Hulk} \\
            \end{align*}
        \pause
        \item Predicados:
            \begin{align*}
                noThor(x) &\equiv (x \neq t) \\
                digno(x) &\equiv x\text{ es digno de levantar el martillo } \\
                levanta(x, y) &\equiv x\text{ levanta a }y \\
            \end{align*}
    \end{itemize}
\end{frame}

\begin{frame}[fragile]
    \frametitle{Un ejemplo}
    \begin{itemize}
        \pause
        \item Premisas:
            \begin{itemize}
                \pause
                \item $\lnot (\exists x: S \mid noThor(x) : digno(x))$
                \pause
                \item $levanta(h, t)$
                \pause
                \item $digno(t)$
            \end{itemize}
        \pause
        \item Conclusión: $digno(h)$
        \pause
        \item Afirmación:
            $$\frac{\lnot (\exists x: S \mid noThor(x) : digno(x)) \land levanta(h, t) \land digno(t)}{digno(h)}$$
        \pause
        \item Demostración: La próxima clase
    \end{itemize}
\end{frame}

\begin{frame}[fragile]
    \frametitle{Algunas reglas}
    \begin{itemize}
        \pause
        \item $\forall$: Para todo, para cada, cualquier, todos
        \pause
        \item $\exists$: Existe, para alguno, algún, hay, al menos uno..
        \pause
        \item $\lnot \forall$: No todos
        \pause
        \item $\lnot \exists$: No existe, para nadie, Ningún, no hay
    \end{itemize}
\end{frame}

\begin{frame}[fragile]
    \frametitle{Ahora ustedes}
    \pause
    Traducir las siguiente afirmación
    \pause
    \begin{center}
        \textit{Todos los animales que ponen huevos son aves o mamíferos. \\
        Todo mamífero que pone huevos es un ornitorrinco. \\
        No hay aves que no tengan plumas. \\
        Perry no tiene plumas pero pone huevos. \\
        Luego, Perry es un ornitorrinco.}
    \end{center}
    \vspace*{165 pt}
\end{frame}

\end{document}
