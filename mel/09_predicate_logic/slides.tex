\documentclass{beamer}

\usepackage[utf8]{inputenc}
\usepackage[english]{babel}

\usepackage{amsmath}
\usepackage{nicefrac}

\usepackage{minted}
\usepackage{fontspec}

\usepackage{xcolor}

\usemintedstyle{friendly}
\setmonofont{Source Code Pro}
\usetheme{metropolis}

\newcommand{\then}{\Rightarrow}

\title{Lógica de predicados}
\author{Matemática estructural y lógica}
\institute{ISIS-1104}
\date{}

\begin{document}

\frame{\titlepage}

\begin{frame}[fragile]
    \frametitle{¿Por qué las proposiciones no bastan?}

    Si tenemos el siguiente enunciado
    \begin{center}
        \textit{Toda la comida italiana es deliciosa.\\ 
        La pizza es comida italiana\\
        Luego la pizza es deliciosa}
    \end{center}
    
    La lógica proposicional es insuficiente para expresar casos particulares y generalizaciones.
\end{frame}

\begin{frame}[fragile]
    \frametitle{Predicados}
        Un predicado sobre $T$ es una función $T \rightarrow Bool$.

        $T$ puede ser cualquier conjunto (tipo) de elementos. Por ejemplo:

        \begin{itemize}
            \item si $k: Int$, $P(k) = k > 0$
            \item si $s:Persona$, $esAdulto(s) = s\text{ es un adulto}$.
        \end{itemize}

        Como son funciones, los predicados no tienen un valor de verdad definido hasta que se evalúan en algún $t:T$.
\end{frame}

\begin{frame}[fragile]
    \frametitle{Predicados}
        Los predicados pueden combinarse usando las operaciones lógicas que ya conocemos.

        Por ejemplo si tenemos los predicados
        $$esAdulto(k) = k\text{ es un adulto}$$
        $$sabeConducir(k) = k\text{ sabe conducir}$$
        Podemos definir un nuevo predicado
        $$esAdulto(k) \land \lnot sabeConducir(k)$$
\end{frame}
\begin{frame}[fragile]
    \frametitle{Predicados}
        Todos los axiomas y reglas de inferencia de lógica proposicional siguen siendo válidos.

        Por ejemplo
        \begin{itemize}
            \item $P(x) \land P(x) \equiv P(x)$
            \item $P(x) \then Q(y) \equiv \lnot P(x) \lor Q(y)$
            \item $\lnot(P(x) \lor Q(x)) \equiv \lnot P(x) \land \lnot Q(x)$
            \item $P(x) \land (P(x) \then Q(x)) \then Q(x)$
        \end{itemize}

        Sin embargo hay que ser cuidadosos con las variables de los predicados
        $$P(x) \then Q(y) \not \equiv P(y) \then Q(x)$$
\end{frame}

\begin{frame}[fragile]
    \frametitle{Constantes}
        En ocasiones, queremos hacer referencia a algún elemento de $T$, para esto usamos constantes.

        Por ejemplo si tenemos el tipo $Animal$ y el predicado
        $$esGato(k) = k\text{ es un gato}$$
        podemos definir una constante $garfield: Animal$ y escribir
        $$esGato(garfield)$$
        para afirmar que la constante $garfield$ cumple el predicado $esGato$
\end{frame}

\begin{frame}[fragile]
    \frametitle{Cuantificador universal}
        Si tenemos el cuantificador 
        $$(\land x:T \mid P(x) : Q(x))$$
        Con $P$ y $Q$ predicados sobre $T$, podemos darle la interpretación
        \begin{center}
            Paras todos los $x$ tales que $P(x)$, se cumple $Q(x)$.
        \end{center}
        Usualmente cambiamos el símbolo $\land$
        $$(\forall x:T \mid P(x) : Q(x))$$
        Por ejemplo
        $$(\forall x:Animal \mid esGato(x) : tieneBigotes(x))$$
        $$(\forall x:Animal \mid esGato(x) : sabeVolar(x))$$
\end{frame}

\begin{frame}[fragile]
    \frametitle{Cuantificador existencial}
        Si tenemos el cuantificador 
        $$(\lor x:T \mid P(x) : Q(x))$$
        Con $P$ y $Q$ predicados sobre $T$, podemos darle la interpretación
        \begin{center}
            Existen $x$ tales que $P(x)$, para los cuales se cumple $Q(x)$.
        \end{center}
        Usualmente cambiamos el símbolo $\land$
        $$(\exists x:T \mid P(x) : Q(x))$$
        Por ejemplo
        $$(\exists x:Animal \mid esGato(x) : comeLasagna(x))$$
        $$(\exists x:Animal \mid esGato(x) : esPerro(x))$$
\end{frame}

\begin{frame}[fragile]
    \frametitle{De vuelta a la pizza}

    Si tomamos $C$ como el tipo de la comida. Y $pizza:C$ como una constante.

    \begin{center}
        \textit{Toda la comida italiana es deliciosa.\\ 
        La pizza es comida italiana.\\
        Luego la pizza es deliciosa.}
    \end{center}
    
    Puede escribirse como

    $$\frac{(\forall c: C \mid esItaliana(c): esDeliciosa(c)),\ esItaliana(pizza)}{esDeliciosa(pizza)}$$
\end{frame}

\begin{frame}[fragile]
    \frametitle{Variables acotadas}
    Es necesario distinguir que variables están libres y cuales están acotadas en éste nuevo lenguaje.

    Por ejemplo:
    \begin{itemize}
        \item $x$ es libre en $x > 5$.
        \item $x$ es libre en $esGato(x)$.
        \item $n$ está acotada en $(\exists n: Int \mid esPar(n) : n > 0)$.
        \item $k$ es libre en $(\forall n: Int \mid esPrimo(n) : n < k)$.
    \end{itemize}

    Entonces
    \begin{itemize}
        \item $x$ está ligada si es la variable sobre la que estoy cuantificando.
        \item $x$ es libre si no estoy cuantificando sobre ella.
    \end{itemize}
\end{frame}

\begin{frame}[fragile]
    \frametitle{Variables acotadas} 
    Si $x$ es libre en una expresión, decimos que depende de $x$.
    \begin{itemize}
        \item $esGato(a)$ depende de $a$.
        \item $(\exists n: Int \mid : n > k)$ depende de $k$.
        \item $(\exists n: Int \mid : n > 0)$ no depende de $n$.
    \end{itemize}
    Cuando una cuantificación depende de alguna variable podemos usarla como una proposición
        $$Q(k) = (\exists n: Int \mid : k = 2n)$$
    entonces
        $$(\forall z: Int \mid esPar(z) : Q(z))$$
    es lo mismo que
        $$(\forall z: Int \mid esPar(z) : (\exists n: Int \mid : z = 2n))$$
\end{frame}

\begin{frame}[fragile]
    \frametitle{Ahora ustedes}
    \begin{itemize}
        \item es $x$ libre en
            $$x + 1 = 5$$
        \item son $a$ y $b$ libres en 
            $$(\exists a: Animal \mid esGato(n) : sonAmigos(a,b))$$
        \item son $x$, $y$, $z$ libres en 
            $$(\forall z: Int \mid z > 0 : (\exists y: Int \mid y > 0 : x^2+y^2=z^2))$$
    \end{itemize}
\end{frame}

\end{document}
